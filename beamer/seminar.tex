\documentclass{beamer}

\usepackage[]{graphicx}

\mode<presentation>
{
  \usetheme{Warsaw}      % or try Darmstadt, Madrid, Warsaw, ...
  \usecolortheme{default} % or try albatross, beaver, crane, ...
  \usefonttheme{default}  % or try serif, structurebold, ...
  \setbeamertemplate{navigation symbols}{}
  \setbeamertemplate{caption}[numbered]
}

\usepackage[slovene]{babel}
\usepackage{bigfoot}
\usepackage{hyperref}
\usepackage[pdf]{graphviz}
\usepackage{forest}
\usepackage{float}
\usepackage{listings}

\title{Make}
\author{Nikola Brković}

\begin{document}

\frame{\titlepage}

\begin{frame}
  \frametitle{Kaj je Make?}

  \begin{itemize}
  \item Orodje, ki avtomatizira proces izgradnje programske opreme
  \item Omogoča opis odvisnosti med izvornimi datotekami v
    konfiguracijski datoteki - Makefilu
  \end{itemize}
\end{frame}

\begin{frame}
  \frametitle{Različice}
  
  \begin{itemize}
  \item GNU, BSD, NMAKE, dmake
  \item Danes se bomo osredotočili na GNU Make
  \item Podpira Linux, Mac OS, Windows, OS/2, DOS..
  \end{itemize}
\end{frame}  

\begin{frame}
  \frametitle{Makefile}

  \begin{itemize}
  \item Vsebuje seznam ciljev
  \item Cilj je odvisen od predpogojev in lahko vsebuje pravila 
  \item Podpira tudi spremenljivke, pogojno izvajanje, itd.
  \end{itemize}
\end{frame}

\begin{frame}[fragile]
  \frametitle{Osnovna sintaksa Makefila}
  
\begin{verbatim}
  cilj [cilj ...]: [predpogoj ...]
   [prvo pravilo]
   ...
   [zadnje pravilo]
\end{verbatim}
\end{frame}

\begin{frame}[t, fragile]
  \frametitle{Primer preprostega Makefila}

\begin{lstlisting}[language=make, frame=single]
all: libfile.a main.o
	ld libfile.a main.o

main.o: main.c
	gcc -c main.c -o main.o

dir.o: dir.c
	gcc -c dir.c -o dir.o

file.o: file.c
	gcc -c file.c -o file.o

libfile.a: dir.o file.o
	ar rcs libfile.a dir.o file.o
\end{lstlisting}
\end{frame}

\begin{frame}[fragile]
  \frametitle{Graf odvisnosti}
  \center{
  \digraph{basicdependency}{
      rankdir=TB;
      size=2;

      dirc [label = "dir.c"]
      diro [label = "dir.o"]
      libfile [label = "libfile.a"]
      dirc -> diro -> libfile
      
      filec [label = "file.c"]
      fileo [label = "file.o"]
            
      mainc [label = "main.c"]
      maino [label = "main.o"]
      
      filec -> fileo -> libfile;
      libfile -> all;
      
      mainc -> maino -> all;
      
      clean;
    }       
  }
\end{frame}

\begin{frame}
  \frametitle{Izvajanje ciljev}
  
  \begin{enumerate}
  \item Če je začetni cilj umeten, in nima nobenih predpogojev, se bo
    izvedel in se postopek konča. (\textit{clean})
  \item Spustimo se po grafu odvisnosti, od vključno trenutnega cilja, v
    nasprotni smeri povezav, dokler ne naletimo na cilj, katerega čas
    zadnje spremembe je starejši kot čas zadnje spremembe enega od
    njegovih predpogojev.
  \item Če ni nobenega zastarelega cilja v celotnem grafu, se postopek
    konča, sicer pa najdeni cilj izvedemo.
  \item Vrnemo se v prejšnji cilj, poskušamo ponoviti korak 2 čim
    večkrat. Preskočimo cilje, ki smo jih morebiti že izvedli. Če
    zastarelih ciljev ne najdemo več, izvedemo trenutni cilj in ponovimo
    ta korak.
  \item Končamo, ko smo prišli nazaj v začetni cilj.
\end{enumerate}
\end{frame}

\begin{frame}[fragile]
  \frametitle{Izvajanje ciljev}
  
  Spremenili smo datoteko file.c:
  \center{
    \digraph{dependencyupdate}{
      rankdir=TB;
      size=2;
      node[style=filled]

      dirc [label="dir.c", fillcolor=white]
      diro [label="dir.o", fillcolor=white]
      dirc -> diro -> libfile
      
      filec [label="file.c", fillcolor=white]
      fileo [label="file.o", fillcolor=lightgrey]
      
      libfile [label="libfile.a", fillcolor=lightgrey]
      filec -> fileo -> libfile
      
      mainc [label="main.c", fillcolor=white]
      maino [label="main.o", fillcolor=white]

      all [fillcolor=lightgrey]
      clean [fillcolor=white]
      
      libfile -> all
      mainc -> maino -> all;

      clean;
    }
  }
\end{frame}

\begin{frame}

\end{frame}

\end{document}

%%% Local Variables:
%%% mode: LaTeX
%%% TeX-master: t
%%% TeX-command-extra-options: "-shell-escape"
%%% End:
