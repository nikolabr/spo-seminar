\documentclass{beamer}

\usepackage[]{graphicx}

\mode<presentation>
{
  \usetheme{Warsaw}      % or try Darmstadt, Madrid, Warsaw, ...
  \usecolortheme{default} % or try albatross, beaver, crane, ...
  \usefonttheme{default}  % or try serif, structurebold, ...
  \setbeamertemplate{navigation symbols}{}
  \setbeamertemplate{caption}[numbered]
}

\usepackage[slovene]{babel}
\usepackage{bigfoot}
\usepackage{hyperref}
\usepackage[pdf]{graphviz}
\usepackage{forest}
\usepackage{float}
\usepackage{listings}

\title{Make}
\author{Nikola Brković}

\begin{document}

\frame{\titlepage}

\begin{frame}
  \frametitle{Kaj je Make?}

  \begin{itemize}
  \item Orodje, ki avtomatizira proces izgradnje programske opreme
  \item Omogoča opis odvisnosti med izvornimi datotekami v
    konfiguracijski datoteki - Makefilu
  \end{itemize}
\end{frame}

\begin{frame}
  \frametitle{Različice}
  
  \begin{itemize}
  \item GNU, BSD, NMAKE, dmake
  \item Danes se bomo osredotočili na GNU Make
  \item Podpira Linux, Mac OS, Windows, OS/2, DOS..
  \end{itemize}
\end{frame}  

\begin{frame}
  \frametitle{Makefile}

  \begin{itemize}
  \item Vsebuje seznam definicij ciljev - pravil
  \item Cilj je odvisen od predpogojev in lahko vsebuje ukaze
  \item Cilji so lahko umetni
  \item Podpira tudi spremenljivke, pogojno izvajanje, itd.
  \end{itemize}
\end{frame}

\begin{frame}[fragile]
  \frametitle{Osnovna sintaksa Makefila}
  
\begin{verbatim}
  cilj [cilj ...]: [predpogoj ...]
   [prvi ukaz]
   ...
   [zadnji ukaz]
\end{verbatim}
\end{frame}

\begin{frame}[t, fragile]
  \frametitle{Primer preprostega Makefila}

\begin{lstlisting}[language=make, frame=single]
all: libfile.a main.o
	ld libfile.a main.o

main.o: main.c
	gcc -c main.c -o main.o

dir.o: dir.c
	gcc -c dir.c -o dir.o

file.o: file.c
	gcc -c file.c -o file.o

libfile.a: dir.o file.o
	ar rcs libfile.a dir.o file.o
\end{lstlisting}
\end{frame}

\begin{frame}[fragile]
  \frametitle{Graf odvisnosti}
  \center{
  \digraph{basicdependency}{
      rankdir=TB;
      size=2;

      dirc [label = "dir.c"]
      diro [label = "dir.o"]
      libfile [label = "libfile.a"]
      dirc -> diro -> libfile
      
      filec [label = "file.c"]
      fileo [label = "file.o"]
            
      mainc [label = "main.c"]
      maino [label = "main.o"]
      
      filec -> fileo -> libfile;
      libfile -> all;
      
      mainc -> maino -> all;
      
      clean;
    }       
  }
\end{frame}

\begin{frame}
  \frametitle{Izvajanje ciljev}
  
  \begin{enumerate}
  \item Če je začetni cilj umeten, in nima nobenih predpogojev, se bo
    izvedel in se postopek konča. (\textit{clean})
  \item Spustimo se po grafu odvisnosti, od vključno trenutnega cilja, v
    nasprotni smeri povezav, dokler ne naletimo na cilj, katerega čas
    zadnje spremembe je starejši kot čas zadnje spremembe enega od
    njegovih predpogojev.
  \item Če ni nobenega zastarelega cilja v celotnem grafu, se postopek
    konča, sicer pa najdeni cilj izvedemo.
  \item Vrnemo se v prejšnji cilj, poskušamo ponoviti korak 2 čim
    večkrat. Preskočimo cilje, ki smo jih morebiti že izvedli. Če
    zastarelih ciljev ne najdemo več, izvedemo trenutni cilj in
    ponovimo ta korak.
  \item Končamo, ko smo prišli nazaj v začetni cilj.
\end{enumerate}
\end{frame}

\begin{frame}[fragile]
  \frametitle{Izvajanje ciljev}
  
  Spremenili smo datoteko file.c:
  \center{
    \digraph{dependencyupdate}{
      rankdir=TB;
      size=2;
      node[style=filled]

      dirc [label="dir.c", fillcolor=white]
      diro [label="dir.o", fillcolor=white]
      dirc -> diro -> libfile
      
      filec [label="file.c", fillcolor=white]
      fileo [label="file.o", fillcolor=lightgrey]
      
      libfile [label="libfile.a", fillcolor=lightgrey]
      filec -> fileo -> libfile
      
      mainc [label="main.c", fillcolor=white]
      maino [label="main.o", fillcolor=white]

      all [fillcolor=lightgrey]
      clean [fillcolor=white]
      
      libfile -> all
      mainc -> maino -> all;

      clean;
    }
  }
\end{frame}

\begin{frame}[fragile]
  \frametitle{Zaporedno izvajanje}

  \begin{itemize}
  \item Izvajanje več poslov hkrati
  \item Poslovne reže (ang. \textit{job slots})
\begin{lstlisting}[frame=single, basicstyle=\sffamily]
make -j 4
\end{lstlisting}
  \item Ključno pri velikih projektih
  \item Primer: Linux (linux-6.13-rc7)
\begin{lstlisting}[frame=single, basicstyle=\sffamily]
$ find . -name '*.c' | wc -l
34788
\end{lstlisting}

  \end{itemize}
\end{frame}

\begin{frame}[fragile]
  \frametitle{Zaporedno izvajanje}
  \begin{columns}
    \begin{column}{0.5\textwidth}
      Pri enem procesu je preprosto, ampak na težave naletimo pri
      rekurzivnih Makih, kjer je potrebna medprocesna
      komunikacija. Rešeno z jobserverjem.
    \end{column}
    \begin{column}{0.5\textwidth}
      \begin{forest}
        for tree={
          draw,
          anchor=north,
          align=center,
        },
        [{make}, align=center,
        [{ld}],
        [{gcc}],
        [{gcc}],
        [{gcc}]]
      \end{forest}
      
      \bigskip
      \bigskip
      
      \begin{forest}
        for tree={
          draw,
          anchor=north,
          align=center,
        },
        [{make}, align=center,
        [{ld}],
        [{gcc}],
        [{make}, [{gcc}], [{gcc}]]]
      \end{forest}
    \end{column}
  \end{columns}

\end{frame}

\begin{frame}
  \frametitle{Jobserver na sistemih POSIX}

  Implementiran z uporabo poimenovane cevi (FIFO), imamo $N$ poslovnih
  rež. Cev vsebuje največ $N - 1$ žetonov.

  Ko želi zagnati posel, mora zasesti režo. Ko se posel konča, pa
  sprosti režo.
  
  \begin{enumerate}
  \item Začetno stanje
    
    \begin{tabular}{ cc|c|cc }
      \cline{2-4}
      $\rightarrow$ & $3$ & $7$ & $2$ & $\rightarrow$ \\
      \cline{2-4}
    \end{tabular}
    
  \item Reža zasedena, prebran žeton
    
    \begin{tabular}{ cc|cc }
      \cline{2-3}
      $\rightarrow$ & $3$ & $7$ & $\rightarrow$ \\
      \cline{2-3}
    \end{tabular}

  \item Reža sproščena, zapisan žeton

    \begin{tabular}{ cc|c|cc }
      \cline{2-4}
      $\rightarrow$ & $2$ & $3$ & $7$ & $\rightarrow$ \\
      \cline{2-4}
    \end{tabular}
    
  \end{enumerate}

  Na starejših sistemih na režo čaka z \textit{read}, na novejših pa \textit{pselect}.
  
  Še podrobneje:
  Paul D. Smith, \url{https://make.mad-scientist.net/papers/jobserver-implementation/}
\end{frame}

\begin{frame}[fragile]
  \frametitle{Jobserver na sistemih POSIX}

\begin{lstlisting}[language=C, basicstyle=\sffamily]
static int job_rfd = -1;

static int
make_job_rfd ()
{
  EINTRLOOP (job_rfd, dup (job_fds[0]));
  // ...
  return job_rfd;
}
\end{lstlisting}
\end{frame}

\begin{frame}[fragile]
  \frametitle{Jobserver na sistemih POSIX}

\begin{lstlisting}[language=C, basicstyle=\sffamily]
void
jobserver_signal ()
{
  if (job_rfd >= 0)
    {
      close (job_rfd);
      job_rfd = -1;
    }
}
\end{lstlisting}
\end{frame}

\begin{frame}[fragile]
  \frametitle{Jobserver na sistemih POSIX}

\begin{lstlisting}[language=C, basicstyle=\sffamily]
unsigned int
jobserver_acquire (int timeout)
{
  // ...
  EINTRLOOP (got_token, read (job_rfd, &intake, 1));
  // ...
}
\end{lstlisting}
\end{frame}

\begin{frame}
  \frametitle{Jobserver na Windowsih}

  Implementiran z uporabo poimenovanega semaforja - sistemskega atomskega števca.

  Semafor se ustvari s funkcijo \textit{CreateSemaphore}.

  Lahko čakamo na proces in semafor hkrati z \textit{WaitForMultipleObjects}.
\end{frame}

\begin{frame}[fragile]
  \frametitle{Jobserver na Windowsih}
\begin{lstlisting}[basicstyle=\footnotesize]
unsigned int
jobserver_setup (int slots, const char *style)
{
  sprintf (jobserver_semaphore_name,
    "gmake_semaphore_%d", _getpid ());

  jobserver_semaphore = CreateSemaphore (
      NULL, /* Use default security descriptor */
      slots, /* Initial count */
      slots, /* Maximum count */
      jobserver_semaphore_name); /* Semaphore name */
\end{lstlisting}
\end{frame}  

\begin{frame}[fragile]
  \frametitle{Jobserver na Windowsih}

\begin{lstlisting}[language=C, basicstyle=\footnotesize]
unsigned int jobserver_acquire (int timeout UNUSED) {
    HANDLE *handles;
    DWORD dwHandleCount;
    DWORD dwEvent;

    handles = xmalloc(process_table_actual_size()
                      * sizeof(HANDLE));

    /* Add jobserver semaphore to first slot. */
    handles[0] = jobserver_semaphore;

    /* Build array of handles to wait for.  */
    dwHandleCount = 1 + process_set_handles (&handles[1]);

    dwEvent = process_wait_for_multiple_objects (
        dwHandleCount,  /* number of objects in array */
        handles,        /* array of objects */
        FALSE,          /* wait for any object */
        INFINITE);      /* wait until object is signalled */

\end{lstlisting}
\end{frame}

\end{document}

%%% Local Variables:
%%% mode: LaTeX
%%% TeX-master: t
%%% TeX-command-extra-options: "-shell-escape"
%%% End:
