% Created 2025-01-12 Sun 23:40
% Intended LaTeX compiler: pdflatex
\documentclass[11pt]{article}
\usepackage[utf8]{inputenc}
\usepackage[T1]{fontenc}
\usepackage{graphicx}
\usepackage{longtable}
\usepackage{wrapfig}
\usepackage{rotating}
\usepackage[normalem]{ulem}
\usepackage{amsmath}
\usepackage{amssymb}
\usepackage{capt-of}
\usepackage{hyperref}
\usepackage[, slovene]{babel}
\usepackage[pdf]{graphviz}
\author{Nikola Brković}
\date{\today}
\title{Make}
\hypersetup{
 pdfauthor={Nikola Brković},
 pdftitle={Make},
 pdfkeywords={},
 pdfsubject={},
 pdfcreator={Emacs 31.0.50 (Org mode 9.8-pre)},
 pdflang={Slovene}}
\begin{document}

\maketitle
\section{Uvod}
\label{sec:orgeb92cae}

\section{Različice}
\label{sec:org9b31356}

Originalno različico programa Make je ustvaril Stuart
Feldman ({Association for Computing Machinery}, 2004). Danes pa poznamo več različic programa Make:

\begin{description}
\item[{GNU Make}] Je danes najbolj razširjena različica Make, privzeta
tako na Linuxu kot tudi MacOS-u. Predstavlja del operacijskega
sistema GNU.

\item[{BSD Make}] Je privzeta različica v operacijskih sistemih družine
BSD (FreeBSD, NetBSD).

\item[{NMAKE (Microsoft Program Maintenance Utility)}] Je različica,
katero je ustvaril Microsoft in teče na Windowsu. Je podprt v
razvojnem okolju Visual Studio.

\item dmake
\end{description}
\section{Upravljanje odvistnosti}
\label{sec:org60e4a87}

\begin{center}
\includegraphics[width=.9\linewidth]{./dot-example.png}
\label{org46323e2}
\end{center}
\section{Vzporedno izvajanje poslov}
\label{sec:org1496189}

\section{Literatura}
\label{sec:orgde4e9e6}

\noindent
{[}1] {Association for Computing Machinery}, \emph{ACM HONORS CREATOR OF LANDMARK SOFTWARE TOOL}, 2004.
\end{document}
