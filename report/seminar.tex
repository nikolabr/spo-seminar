\documentclass[notitlepage]{report}

\usepackage{titling}
\usepackage[pdftex]{graphicx}
\usepackage[slovene]{babel}

\bibliographystyle{plain}

\title{%
  Make \\
  \large Poročilo seminarske naloge
}
\author{Nikola Brković}

\begin{document}
\maketitle

\begin{abstract}
\end{abstract}

\section*{Uvod}

\section*{Različice}

Originalno različico programa Make je ustvaril Stuart Feldman
\cite{acm-award}. Danes pa poznamo več različic programa Make:

\begin{description}
\item[GNU Make] Je danes najbolj razširjena različica Make, privzeta
  tako na Linuxu kot tudi MacOS-u. Predstavlja del operacijskega
  sistema GNU.
  
\item[BSD Make] Je privzeta različica v operacijskih sistemih družine
  BSD (FreeBSD, NetBSD).

\item[NMAKE (Microsoft Program Maintenance Utility)] Je različica,
  katero je ustvaril Microsoft in teče na Windowsu. Je podprta v
  razvojnem okolju Visual Studio.
\end{description}
  
\section*{Upravljanje odvisnosti}

\section*{Vzporedno izvajanje poslov}

\bibliography{seminar}

\end{document}

%%% Local Variables:
%%% mode: LaTeX
%%% TeX-master: t
%%% End:
