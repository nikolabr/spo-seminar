\documentclass[notitlepage]{report}

\usepackage{titling}
\usepackage[pdftex]{graphicx}
\usepackage[slovene]{babel}
\usepackage{bigfoot}
\usepackage{hyperref}
\usepackage[pdf]{graphviz}
\usepackage{forest}
\usepackage{float}
\usepackage{listings}

\bibliographystyle{acm}

\title{%
  Make \\
  \large Poročilo seminarske naloge
}
\author{Nikola Brković}


\lstset{language=C,
  keywordstyle=\color{blue}\bfseries,
  commentstyle=\color{green},
  stringstyle=\ttfamily\color{red!50!brown},
  showstringspaces=false}

\begin{document}
\maketitle

\begin{abstract}
  Make je orodje za avtomatizacijo izgradnje programske opreme. To
  poročilo ponuja pregled nad osnovnimi funkcionalnostimi in koncepti
  delovanja, ter podrobnosti različice GNU Make.
\end{abstract}

\section*{Uvod}

Izgradnja programske opreme je pri velikih projektih lahko zelo
kompleksen in zamuden postopek. Postopek pretvorbe izvorne kode v neko
končno obliko (npr. izvršljivo datoteko) najpogosteje poteka v več
fazah. Z orodjem Make lahko postopek izgradnje opišemo v datoteki, ki
se imenuje Makefile, kjer opišemo, kako določeno fazo izvesti in
odvisnosti med posameznimi fazami.

\section*{Različice}

Originalno različico programa Make je ustvaril Stuart Feldman v 70-ih
letih \cite{acm-award}, in je takrat delovala na zgodnjih različicah
operacijskega sistema Unix. Danes pa poznamo več različic programa
Make:

\begin{description}
\item[GNU Make] Je danes najbolj razširjena različica Make, privzeta
  tako na Linuxu kot tudi MacOS-u. Predstavlja del operacijskega
  sistema GNU. Deluje pa tudi na Windowsu, zlasti se uporablja v
  unixaških okoljih na Windowsu (Cygwin/MSYS2).
  
\item[BSD Make] Je privzeta različica v operacijskih sistemih družine
  BSD (FreeBSD, NetBSD).

\item[NMAKE (Microsoft Program Maintenance Utility)] Je različica,
  katero je ustvaril Microsoft in teče na Windowsu. Je podprta v
  razvojnem okolju Visual Studio. \cite{ms-nmake}

\item[dmake (Distributed Make)] Je različica, katero razvija Apache Software Foundation,
  in se uporablja pri projektu OpenOffice \cite{dmake}. Predhodno jo je razvijalo
  podjetje Sun.
  
\end{description}

Ker je to najbolj priljubljena različica, se bom v tem poročilu
osredotočil na GNUjevsko različico, natančneje na izdajo 4.4.1. Kosi
izvorne kode in podrobnosti implentacije bodo izvirali iz te
različice, vendar splošni koncepti velja tudi za druge različice.

\section*{Osnovna sintaksa}

Makefile (poleg drugih zadev, kot so definicije spremenljivk) v osnovi
sestavljajo definicije ciljev. Definicija enega (ali več enakih
ciljev) izgleda tako:

\begin{verbatim}
  cilj [cilj ...]: [predpogoj ...]
   [prvo pravilo]
   ...
   [zadnje pravilo]
\end{verbatim}

V večini primerov predstavlja ime cilja relativno pot (od Makefilea)
do datoteke (ali mape). Takim ciljem pravimo tudi ciljne datoteke
(ang. \textit{target file}). Ciljem, ki ne predstavljajo datoteke (ali
mape), pa pravimo umetni cilji (ang. \textit{phony targets}).

Predpogoji (oziroma odvisnosti) so drugi cilji, od katerih je
definirani cilj odvisen, kar pomeni, da se morajo uspešno izvesti,
preden se definirani cilj izvede.

Vsako pravilo predstavlja stavek, ki se lahko izvede v sistemski
lupini. Ker GNU Make izvira iz unixaškega sveta, je razvit z
unixaškimi lupini v mislih, ampak podpira tudi neunixaške lupine, kot
so \verb|cmd.exe| (Windows NT), \verb|command.com| (DOS, Windows 95 in
98), \verb|4OS2| (OS/2). V stavek lahko vključimo določene izraze,
katere bo Make izračunal ob izvajanju, ampak je sintaksa samih pravil
precej odvisna od operacijskega sistema in izbrane sistemske lupine.

\section*{Odvisnosti}

V osnovi predstavlja Makefile tekstovno reprezentacijo acikličnega
usmerjenega grafa odvisnosti. Vsak cilj predstavlja eno vozlišče v
grafu, odvisnosti so pa usmerjene povezave.

Ko Make zaženemo, lahko podamo cilj, ki naj bi se izvedel, ali pa
lahko nastavimo privzeti cilj v Makefileu. V nadaljevanju bomo temu
cilju pravili začetni cilj. Ob vsakem zagonu, Make bo izvedel samo
tiste cilje v grafu odvisnosti, ki jih je potrebno izvesti.

Da Make ugotovi, ali je nek cilj potrebno ponovno izvesti, se zanaša
na čas spremembe datoteke (\textit{mtime}). V GNU Make je razreševanje
odvisnosti (\textit{dependency engine}) implementirano v datoteki
\textit{src/remake.c}. Za enostavne Makefile deluje algoritem tako:

\begin{enumerate}
\item Če je cilj umeten, in nima nobenih predpogojev, se bo ta cilj
  izvedel, in se postopek konča.
\item Spustimo se po grafu odvisnosti, od vključno začetnega cilja, v
  nasprotni smeri povezav, dokler ne naletimo na cilj, katerega čas
  zadnje spremembe je starejši kot čas zadnje spremembe enega od
  njegovih predpogojev.
\item Če takšnega cilja ni, potem ne izvedemo
  ničesar, in se postopek konča.
\end{enumerate}

\begin{figure}[H]
  \begin{center}
    \digraph{basicdependency}{
      rankdir=TB;
      size=2;
      
      filec [label = "file.c"]
      fileo [label = "file.o"]
      
      mainc [label = "main.c"]
      maino [label = "main.o"]
      
      filec -> fileo -> all;
      mainc -> maino -> all;

      clean;
    }    
  \caption{Primer grafa odvistnosti za preprosti Makefile. ``all'' pa
    ``clean'' sta umetna cilja, ostali so datoteke.}
\end{center}
\end{figure}

\begin{figure}[H]
  \begin{center}
    \digraph{dependencyupdate}{
      rankdir=TB;
      size=2;
      node[style=filled]
      
      filec [label="file.c", fillcolor=white]
      fileo [label="file.o", fillcolor=white]
      
      mainc [label="main.c", fillcolor=lightgrey]
      maino [label="main.o", fillcolor=lightgrey]

      all [fillcolor=lightgrey]
      clean [fillcolor=white]
      
      filec -> fileo -> all;
      mainc -> maino -> all;

      clean;
  }
  \caption{Če posodobimo datoteko main.c, bodo izvedeni cilji, označeni s
    sivo barvo.}
  \end{center}
\end{figure}

\section*{Vzporedno izvajanje poslov}

Pri izvajanju programa Make, mu lahko podamo dodatno zastavico
\verb|-j| (npr. \verb|make -j 32|), ki nastavi največje število
poslov, ki jih Make lahko izvaja hkrati. Temu število pravimo tudi
število poslovnih rež (ang. \textit{job slots}).

\begin{figure}[H]
  \begin{center}
    \begin{forest}
      for tree={
        draw,
        anchor=north,
        align=center,
      },
      [{make}, align=center,
      [{ld}],
      [{gcc}],
      [{gcc}],
      [{gcc}]]
    \end{forest}
    \caption{Primer izvajanja Make s štirimi poslovnimi režami}
  \end{center}
\end{figure}

Težave pri vzporednem izvajanju poslov povzroča rekurzivna uporaba
Make (ang. \textit{recursive make}). Rekurzivna uporaba pomeni, da se
znotraj nekega pravila v Makefilu ponovno kliče Make. Čeprav je takšna
uporaba odsvetovana (``Recursive Make Considered Harmful'')
\cite{Miller2008RecursiveMC}, se še vedno uporablja pri večjih
projektih, ki so sestavljeni iz več datotek z izvorno kodo.

\begin{figure}[H]
  \begin{center}
    \begin{forest}
      for tree={
        draw,
        anchor=north,
        align=center,
      },
      [{make}, align=center,
      [{ld}],
      [{gcc}],
      [{make}, [{gcc}], [{gcc}]]]
    \end{forest}
    \caption{Primer izvajanja rekurzivnega Make s štirimi poslovnimi režami}
  \end{center}
\end{figure}

\begin{figure}[H]
  \begin{lstlisting}[language=make, basicstyle=\sffamily]
    subsystem:
        cd subdir && $(MAKE)
  \end{lstlisting}
  \caption{Primer rekurzivnega izvajanja Make}
\end{figure}

V primeru, ko na enem sistemu hkrati teče več procesov Make, je težko
spremljati, koliko poslov se dejansko izvaja v vseh procesih
skupaj. Pri GNU Make so ta problem rešili s komponento, ki se imenuje
\textit{jobserver} in temelji na medprocesni komunikaciji (IPC)
\cite{jobserver-impl}.

Implementacija jobserverja se razlikuje glede na operacijski
sistem. Na sistemih, ki ustrezajo standardu POSIX, je jobserver
implementiran z uporabo poimenovane cevi (ang. \textit{named pipe}, v
unixaški terminologiji tudi \textit{FIFO}). Na sistemih POSIX, ki ne
podpirajo poimenovanih cevi, mora starš podati otroškim procesom opisnik
datoteke. Deluje pa na preprostem principu:

\begin{enumerate}
\item V primeru, ko imamo $N$ poslovnih rež, bo korenska instanca Make
  (tista, ki ni klicana rekurzivno), v cev zapisala $N - 1$
  žetonov. Žeton je predstavljen z enim bajtom. Nima vnaprej določene
  vrednosti, vsi žetoni so lahko popolnoma enaki.

  \begin{figure}[H]
    \begin{center}
      \begin{tabular}{ cc|c|c|cc }
        \cline{2-5}
        $\rightarrow$ & $3$ & $7$ & $2$ & $6$ & $\rightarrow$ \\
        \cline{2-5}
      \end{tabular}
      \end{center}
    \caption{Začetno stanje cevi}
  \end{figure}
\item Ko želimo zagnati novi proces, moramo ``zasesti režo'' oziroma
  prebrati en žeton iz cevi. Načeloma želimo, da bo to branje
  blokirajoče, kar pomeni da se iz sistemskega klica \verb|read| ne
  bomo vrnili, dokler ne bo žeton na voljo. Implementirano v funkciji
  \verb|jobserver_acquire| v izvorni datoteki \verb|src/posixos.c|.
  
  \begin{figure}[H]
    \begin{center}
      \begin{tabular}{ cc|c|cc }
        \cline{2-4}
        $\rightarrow$ & $3$ & $7$ & $2$ & $\rightarrow$ \\
        \cline{2-4}
      \end{tabular}
      \end{center}
    \caption{Stanje cevi po prebranem žetonu}
  \end{figure}

\item Ko smo posel končali, moramo pa ``spustiti režo'' oziroma
  zapisati žeton nazaj v cev. Pri tem pa mora vrednost zapisanega žetona biti
  enaka tisti, ki smo jo prebrali, ko smo posel ustvarili. Implementirano v funkciji
  \verb|jobserver_release| v izvorni datoteki \verb|src/posixos.c|.

    \begin{figure}[H]
    \begin{center}
      \begin{tabular}{ cc|c|c|cc }
        \cline{2-5}
        $\rightarrow$ & $6$ & $3$ & $7$ & $2$ & $\rightarrow$ \\
        \cline{2-5}
      \end{tabular}
      \end{center}
    \caption{Stanje cevi po zapisanem žetonu}
  \end{figure}
  
\end{enumerate}

Vendar, pri takšni implementaciji obstaja ena pomanjkljivost. Če smo v
situaciji, kjer želimo zagnati novi posel, lahko čakamo, da se bo
sprostila reža pri jobserverju. Med tem čakanjem se lahko zgodi, da se
konča nek posel, katerega je ustvaril trenutni proces Makea. Če je
čakanje na režo v sistemskem klicu \verb|read| blokirajoče, potem se
proces ne more ustrezno odzvati na signal \verb|SIGCHLD|, ki sporoči,
da se je otroški proces posla končal. \footnote{Na to omejitev sicer
  ne bi naleteli, če bi imeli proces z več nitmi, ampak je GNU Make
  zaradi prenosljivosti popolnoma eno-niten program.}

Pri starejših sistemih POSIX, je Make to težavo respil tako, da,
preden začne čakati na žeton, poleg že obstoječega opisnika bralnega
konca cevi, ustvari eno kopijo opisnika, na katero čaka, ko želi
zasesti režo. Ko otrok umre, bo rokovalnik signala \verb|SIGCHLD|
zaprl kopijo, in s tem bo se bo vrnil iz sistemskega klica.

\begin{lstlisting}[language=C, basicstyle=\sffamily]
static int job_rfd = -1;

static int
make_job_rfd ()
{
  EINTRLOOP (job_rfd, dup (job_fds[0]));
  // ...
  return job_rfd;
}

void
jobserver_signal ()
{
  if (job_rfd >= 0)
    {
      close (job_rfd);
      job_rfd = -1;
    }
}

unsigned int
jobserver_acquire (int timeout)
{
  // ...
  EINTRLOOP (got_token, read (job_rfd, &intake, 1));
  // ...
}
\end{lstlisting}

Novejše različice Mac OS-a in Linuxa pa ponujajo sistemski klic, ki se
imenuje \verb|pselect|. \verb|pselect| ponuja možnost čakanja na enega
ali več datotečnih opisnikov in je lahko prekinjen s strani
signala. Novejše različice GNU Make uporabljajo ta sistemski klic.

Na operacijskih sistemih Windows je pa jobserver implementiran
popolnoma drugače. Od Windowsa XP naprej, Windows ponuja poimenovane
semaforje (ang. \textit{named semaphore}), ki predstavljajo atomski
števec, do katerega lahko dostopajo vsi procesi na
sistemu. Poimenovani semafor ustvarimo s funkcijo
\verb|CreateSemaphore|, pri čemer moramo podati ime, pod katerim bo
semafor na voljo ostalim procesom \cite{ms-semaphore}.

\begin{lstlisting}[language=C, basicstyle=\sffamily]
unsigned int
jobserver_setup (int slots, const char *style)
{
  // ...
  sprintf (jobserver_semaphore_name, "gmake_semaphore_%d",
           _getpid ());

  jobserver_semaphore = CreateSemaphore (
      NULL,          /* Use default security descriptor */
      slots,                          /* Initial count */
      slots,                          /* Maximum count */
      jobserver_semaphore_name);      /* Semaphore name */
      
   // ...

   return 1;   
}      
\end{lstlisting}

Windows ponuja sistemsko funkcijo \verb|WaitForMultipleObjects| iz
knjižnice \verb|Kernel32.dll|, podobno unixaški funkciji
\verb|select|, ki omogoča čakanje na več datotečnih opisnikov. Vendar,
za razliko od \verb|select|, ki podpira samo datotečne opisnike,
\verb|WaitForMultipleObjects| omogoča čakanje tudi na druge vire
operacijskega sistema, vključno s procesi. Zaradi tega nimamo težav s
hkratnim čakanjem na proces in semafor. \footnote{Starejše različice
  Windowsa sploh ne ponujajo mehanizma, ki bi bil enakovreden
  unixaškemu signalu \verb|SIGCHLD|, kjer bi OS proces obvestil o
  koncu otroškega procesa.}

\bibliography{seminar}

\end{document}

%%% Local Variables:
%%% mode: LaTeX
%%% TeX-master: t
%%% TeX-command-extra-options: "-shell-escape"
%%% End:
